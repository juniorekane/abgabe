%%%%%%%%%%%%%%%%%%%%%%%%%%%%%%%%%%%%%%%%%%%%%%%%%%%%%%%%%%%%%%%%%%%
%                                                                 %
%                 Packages / Grundeinstellungen                   %
%                                                                 %
%%%%%%%%%%%%%%%%%%%%%%%%%%%%%%%%%%%%%%%%%%%%%%%%%%%%%%%%%%%%%%%%%%%


% Festlegung des Allgemeinen Dokumentenformat
\documentclass[a4paper,12pt,parskip=half,headsepline,DIV=12,numbers=noenddot]{scrartcl}

%%%%%% Muss in die documentclass %%%%%%%%
%BCOR=12mm, Korrektur fuer die Bindung
%DIV12 DIV-Wert fuer die Erstellung des Satzspiegels

% Keine floats in andere Sections
\usepackage[section]{placeins}

% Weitere Pakete
\usepackage{microtype}
\usepackage{caption}
\usepackage{fontspec}
\usepackage{pdflscape}
\usepackage{float}
\usepackage{dirtree}
\usepackage{subcaption}
\usepackage{enumitem}

% Booktabs Tabellen
\usepackage{tabularray}
\UseTblrLibrary{booktabs}
\DefTblrTemplate{contfoot-text}{normal}{Fortsetzung auf nächster Seite}
\SetTblrTemplate{contfoot-text}{normal}
\DefTblrTemplate{conthead-text}{normal}{}
\SetTblrTemplate{conthead-text}{normal}

% Um Captions in Tabellen zu deaktivieren 
%\DefTblrTemplate{caption-tag}{default}{}
%\DefTblrTemplate{caption-sep}{default}{}
%\DefTblrTemplate{caption-text}{default}{}

% Grafiken aus PNG Dateien einbinden
\usepackage{graphicx}

% Deutsche Sonderzeichen und Silbentrennung nutzen
\usepackage[ngerman]{babel}
\usepackage{blindtext}

% Eurozeichen einbinden
\usepackage[right]{eurosym}

% Kopf- und Fußzeilen
\usepackage[headsepline,autooneside=false]{scrlayer-scrpage}
\clearpairofpagestyles

% Schriftart 
\usepackage{lmodern}
\setmainfont{TeX Gyre Termes}
\setsansfont{TeX Gyre Adventor}

% Floatende Bilder ermöglichen
\usepackage{floatflt}

% tikz
\usepackage{tikz}
\usetikzlibrary{calc,arrows,math}
\usetikzlibrary{shapes.geometric,positioning}

%% Schaltpläne nach europäischen Richtlinien
\usepackage[european]{circuitikz}
\tikzset{x=1mm,y=1mm}

\usepackage{siunitx}
\sisetup{output-decimal-marker={,},detect-all}

% Bricht lange URLs "schön" um
\usepackage[hyphens,obeyspaces,spaces]{url}

% Paket für Textfarben
\usepackage{xcolor} 
\definecolor{LightGray}{gray}{0.9}
\usepackage[pagecolor=white]{pagecolor}

% Mathematische Symbole importieren
\usepackage{amssymb}

% Paket für Zeilenabstand
\usepackage{setspace}

% Für Bildbezeichner
\usepackage{capt-of}

% Für Stichwortverzeichnis
\usepackage{makeidx}

% Für if und while 
\usepackage{etoolbox}

% Konfiguriere das Inhaltsverzeichnis
\usepackage{tocbasic}
\DeclareTOCStyleEntries[
  raggedentrytext,
  %numwidth=0pt, if numbers=noenddot is not set
  numsep=1ex,
  dynnumwidth,
]{tocline}{chapter,section}
\DeclareTOCStyleEntries[
  linefill=\TOCLineLeaderFill,
]{tocline}{section,subsection,subsubsection,paragraph,subparagraph}

\newcommand*\tocentryformat[1]{{\rmfamily#1}}
\RedeclareSectionCommands
  [tocentryformat=\tocentryformat,tocpagenumberformat=\tocentryformat]{subsection,subsubsection,paragraph,subparagraph}

\newcommand*\tocentrysectionformat[1]{{\rmfamily\bfseries#1}}
\RedeclareSectionCommands
  [tocentryformat=\tocentrysectionformat,tocpagenumberformat=\tocentrysectionformat]{section}  
  
\DeclareTOCStyleEntries[
  pagenumberbox=\hbox,
  dynnumwidth]{tocline}{chapter,section,subsection,subsubsection,paragraph,subparagraph,figure,table}

% Für schönere Listings
\usepackage[outputdir=log, newfloat,]{minted}
\setminted{
  frame=lines,
  framesep=2mm,
  baselinestretch=1.2,
  bgcolor=LightGray,
  fontsize=\footnotesize,
  linenos,
  breaklines=true,
  breakanywhere=true,
  autogobble,
  tabsize=2
}
\setmintedinline{}

% Keine Floats bei Listings
\newenvironment{code}[2]
  {\captionsetup{type=listing}
  \providecommand{\captiontitle}{#1}
  \providecommand{\labeltitle}{#2}
  \vspace*{0.3cm}
  }
  {
  \vspace*{-0.8cm}
  \caption{\captiontitle}
  \label{\labeltitle}
  \vspace*{0.35cm}
  }
\SetupFloatingEnvironment{listing}{}

% Nummerierung inkl. Section
\usepackage{chngcntr}
\counterwithin{table}{section}
\counterwithin{figure}{section}
\counterwithin{listing}{section}

% Abkürzungsverzeichnis
\usepackage[printonlyused, smaller, withpage]{acronym}

% Erzeugt Inhaltsverzeichnis mit Querverweisen zu den Abschnitten (PDF Version)
\usepackage[bookmarksnumbered,hyperfootnotes=false,hypertexnames=false]{hyperref}
\hypersetup{
    colorlinks=true,
    linkcolor=black,
    filecolor=blue,
    citecolor = black,      
    urlcolor=blue,
}

% Darf erst hier eingebunden werden! 
\usepackage{subfiles}
\usepackage{csquotes}

% Indexerstellung
\makeindex


%%%%%%%%%%%%%%%%%%%%%%%%%%%%%%%%%%%%%%%%%%%%%%%%%%%%%%%%%%%%%%%%%%%
%                                                                 %                    
%                   Definition Zitierstil                         %
%                                                                 %
%%%%%%%%%%%%%%%%%%%%%%%%%%%%%%%%%%%%%%%%%%%%%%%%%%%%%%%%%%%%%%%%%%%

% Zitierung nach IEEE

\usepackage[
backend=biber,
style=ieee,
autocite=inline,
]{biblatex}
\addbibresource{bibtex/hauptdatei.bib}

% Zitierung nach APA

%\usepackage[
%backend=biber,
%style=apa,
%autocite=inline,
%]{biblatex}
%\addbibresource{bibtex/hauptdatei.bib}

\setcounter{biburllcpenalty}{7000}
\setcounter{biburlucpenalty}{8000}
%%%%%%%%%%%%%%%%%%%%%%%%%%%%%%%%%%%%%%%%%%%%%%%%%%%%%%%%%%%%%%%%%%%
%                                                                 %                    
%                    Definition Deckblatt                         %
%                                                                 %
%%%%%%%%%%%%%%%%%%%%%%%%%%%%%%%%%%%%%%%%%%%%%%%%%%%%%%%%%%%%%%%%%%%

% true für Bachelorarbeit / false für Hausarbeit
\newbool{bachelorarbeit}
\setbool{bachelorarbeit}{false}

% Setze Fachbereich
\newcommand{\department}{Fachbereich II \\ Management und Informationssysteme}

% Setze Studiengang
\newcommand{\studyprogram}{Wirtschaftsinformatik B.Sc.}

% Setze Modulname (bachelorarbeit muss false sein)
\newcommand{\modulname}{Vernetzte Systeme}

% Setze Dozent:in (bachelorarbeit muss false sein)
\newcommand{\auditor}{\textbf{Dozent:} \> Prof. Dr. Oliver Radfelder}

% Setze Gutachter:innen (bachelorarbeit muss true sein)
\newcommand{\firstauditor}{\textbf{Erstgutachter:} \> Prof. Dr. Oliver Radfelder}
\newcommand{\secondauditor}{\textbf{Dozentin:} \> Prof. Dr. Karin Vosseberg}

% Setze Titel und Untertitel der Abreit 
\newcommand{\thetitle}{Analyse verschiedener Dockerkonstellationen}

% Setze Autor:in und MatNr.
\newcommand{\theauthor}{Junior Ekane}
\newcommand{\matriculationnumber}{40128}

% Abstand zwischen Name und MatNr. (siehe Deckblatt)
\newcommand{\myspace}{1.0cm}

% Muss in src/basic_structure/deckblatt.tex einkommentiert werden! 

\newcommand{\secondauthor}{\> Florian Quaas \> MatNr. 39952\\}
\newcommand{\thirdauthor}{\> Herman Tsago \> MatNr. 00000\\}
\newcommand{\fourthauthor}{\> Steve Aguiwo II \> MatNr. 40088\\}

% PDF Metadaten
%\hypersetup{pdfinfo={
%Title={\thetitle},
%Author={\theauthor}
%}}

\hypersetup{pdfinfo={
Title={\thetitle},
Author={\theauthor}
}}

%%%%%%%%%%%%%%%%%%%%%%%%%%%%%%%%%%%%%%%%%%%%%%%%%%%%%%%%%%%%%%%%%%%
%                                                                 %                    
%                     Beginn des Inhalts                          %
%                                                                 %
%%%%%%%%%%%%%%%%%%%%%%%%%%%%%%%%%%%%%%%%%%%%%%%%%%%%%%%%%%%%%%%%%%%

%%%%%%%%%%%%%%%%%%%%%%%%%%%%%%%%%%%%%%%%%%%%%%%%%%%%%%%%%%%%%%%%%%%
%  Special Characters:                                            %
%                                                                 %
%             \& \% \$ \# \_ \{ \}                                %
%             \textasciitilde (~)                                 %
%             \textasciicircum (^)                                %     
%             \textbackslash (\)                                  %                    
%      \glqq Text\grqq{} für Anführungszeichen                    %
%%%%%%%%%%%%%%%%%%%%%%%%%%%%%%%%%%%%%%%%%%%%%%%%%%%%%%%%%%%%%%%%%%%

\begin{document}



% Definition Header Sections sollen in der Kopfzeile stehen; Kopfzeile mit Unterstrich
\automark[subsection]{section}
\KOMAoptions{headsepline=true}
%\ihead{Kopfzeile innen}
%\chead{Kopfzeile außen}
\ohead{\headmark}

% Definition footer \pagemark steht für Seitennummer
%\ifoot{Fußzeile innen}
%\cfoot{Fußzeile Mitte}
\ofoot{\pagemark}

% Hier werden die Trennvorschläge inkludiert
%\input{src/basic_structure/trennung.tex}


% Titelseite 
\input{src/basic_structure/deckblatt.tex}
\newpage

% Singlespacing (Zeilenabstand) (Default)
\singlespacing
\normalsize

% Abstract falls gewünscht
%\thispagestyle{empty}
%\input{abstract}
%\newpage

% Inhaltsverzeichnis anzeigen
\pagestyle{empty}
\tableofcontents
\newpage
\pagestyle{headings}

% Header für den Inhalt 
\KOMAoptions{headsepline=true}
\ohead{\headmark}

% Input Inhalt
\section{Einleitung}\label{einleitung}

\subsection{Vorstellung des Moduls Vernetzte Systeme}

\section{Projektablauf}\label{Projektablauf}

\subsection{Vorstellung der Projektidee}

\subsection{Projektorganisation und Teamarbeit}


\section{Werkzeuge}\label{Werkezuge}


\subsection{Verwendete Werkzeuge}

Bei so einem Projekt, da wo es draum geht mit mehreren entfernten Systemen zu arbeiten, braucht man genauso wi ebei anderen Projekten auch Werkzeuge, mit denen man seinen Entwurf auf die Beine bringen kann.\\

Hierfür hatten wir schon aus der Vorlesung einige Vorgaben bekommen wie zum Beispiel die Verwendung einer Docker-Umgebung, woran wir uns auch entsprechend eingehalten haben. aber damit unser Entwurf unserer Vorstellung entspricht haben wir auch noch andere Werkzeuge gebraucht, damit wir zum Beispiel unsere mit der Programmiersprache Java entwickelte Mini-Anwendung laufen lassen können, Dateneingaben speichern können, das Einkommen von mehreren Anfragen zu gleicher Zeit verwalten können und auch dann das Verhalten unserer Mini-Anwendung beobachten können. Hier sind unsere Werkzeuge aufgelistet:

\begin{itemize}
\item \textbf{Git}: Es ist tatsächlich so, dass man oft alleine entwickelt, wenn es sich um solche minimale Anwendungen wie bei uns handelt, doch ist es aber auch nicht ungewöhnlich, dass man mit anderer Leuten kooperieren muss und wenn dies der Fall ist, muss auf irgendeine Art sichergestellt werden, dass alle Leute auf einem gemeinsamen Ordner arbeiten, dessen Inhalt jeder verwalten und auch nach Bedarf zurücksetzen kann. Hierzu ist Git das perfekte Werkzeuge, denn es ermöglicht eine Versionsverwaltung von Dateien, sodass man auch auf einen vergangenen Inhalt zugreifen kann, wenn gemerkt wird, dass etwas schiefgelaufen ist oder dass man etwas gelöscht hat, dass im code von großem Belang war.

\item \textbf{Apache Tomcat}: Es handelt sich hierbei um eine Open-Source-Implemtierung verschiedener Spezifikationen wie etwa der \textbf{jakarta Servlet}, \textbf{jakarta Anmerkungen}, \textbf{jakarta Authentifizierung} und noch mehr. Es gehört die Apache Software Foundation. Mit einem Tomcat-Server wird möglich eine Anwendung, die ursprünglich in der Programmiersprache Java geschrieben wurde, als Webanwendung laufen zu lassen, was bei uns tatsächlich erfoderlich war.

\item \textbf{Docker}: Es ist ein revolutionäres Tool, das die Art und Weise, wie Anwendungen entwickelt, ausgeliefert und ausgeführt werden, verändert hat. Es ermöglicht Entwicklern, ihre Anwendungen samt Abhängigkeiten in sogenannten Containern zu verpacken. Diese Containerisierung erleichtert die Bereitstellung und den Betrieb von Anwendungen, da sie auf jedem System laufen können, das Docker unterstützt, unabhängig von der Umgebung. Dies sorgt für Konsistenz über Entwicklung, Test und Produktion hinweg und vermeidet das berüchtigte \/"Aber bei mir lokal funktioniert es"\/ Problem. Für unser Projekt ist Docker essentiell, da es uns ermöglicht, verschiedene Konstellationen mit MariaDB, Redis und anderen Tools effizient zu orchestrieren und zu analysieren.

\item \textbf{MariaDB}: Dieses Werkzeug muss keiner mehr vorgestellt werden, der sich schon zumindest einmal mit Datenbanken auseinander gesetzt hat. MariaDB ist eine sehr berühmte Open-Source-Datenbank, die als Abspaltung von MySQL entstanden ist, mit dem Ziel, vollständige Open-Source-Freiheit zu gewährleisten. Sie bietet eine hohe Kompatibilität mit MySQL, was bedeutet, dass Anwendungen, die für MySQL geschrieben sind, in der Regel ohne Änderungen mit MariaDB funktionieren. In unserem Projekt nutzen wir MariaDB zur Speicherung von Daten, die von unserer Servlet-Anwendung verwendet werden.

\item \textbf{Redis}: Es geht hierbei um ein in-memory Datenstrukturspeicher also eine NoSQL-Datenbank, der als Datenbank, Cache und Message Broker verwendet werden kann. Es unterstützt Datentypen wie Strings, Hashes, Listen, Sets und mehr, was es sehr flexibel macht. Redis ist bekannt für seine hohe Geschwindigkeit und Effizienz bei Operationen mit Daten im Speicher. Bei uns hatten wir die Idee gehabt, dass wir mit Redis eine Warteschlange bauen können. redis kennen wir schon zusammen mit MariaDB und K6 aus dem zweiten Semester, in dem wir eine kurze Einführung in jeden Themen gehabt haben.

\item \textbf{Tcpdump}: Aus der vorletzten veranstaltung kennengelernt ist Tcpdump ein mächtiges Kommandozeilen-Tool zur Netzwerküberwachung und -analyse. Es ermöglicht es, den Datenverkehr auf einem Netzwerk zu erfassen und zu analysieren, was für die Diagnose von Netzwerkproblemen oder für Sicherheitsanalysen unerlässlich ist. In unserem Projekt verwenden wir tcpdump, um den Datenverkehr zwischen den verschiedenen Containern zu analysieren und zu verstehen, wie unsere Anwendungen kommunizieren und wie sich die Netzwerkleistung auf die Gesamtleistung der Anwendung auswirkt.

\item \textbf{Ping}: Sehr beghert ist Ping ein einfaches, aber äußerst nützliches Netzwerk-Utility, das verwendet wird, um die Verfügbarkeit und die Latenz (Ping-Zeit) zwischen zwei Netzwerkknoten zu testen. Es sendet ICMP "Echo Request" Nachrichten an das Ziel und erwartet "Echo Reply" Nachrichten zurück. In unserem Docker-basierten Setup nutzen wir Ping, um die Netzwerkverbindung zwischen den Containern sowie die Verbindung zur Außenwelt zu überprüfen, was für die Fehlersuche und Leistungsoptimierung unerlässlich ist.

\item \textbf{K6}: Es ist ein modernes, Open-Source-Performance-Testing-Tool, das für Entwickler konzipiert wurde, um die Leistung und Skalierbarkeit von Anwendungen in einer DevOps-Umgebung zu testen. Es ermöglicht das Schreiben von Tests in JavaScript, was die Erstellung realistischer Test-Szenarien vereinfacht. In unserem Projekt verwenden wir K6, um die Belastbarkeit und Performance unserer Dockerkonstellationen unter verschiedenen Bedingungen zu testen, um sicherzustellen, dass unsere Anwendungen auch unter Last zuverlässig und effizient laufen. Effektiv wurde K6 für die Testphase bei uns verwendet. Mehr dazu unten bei Lasttest.

\end{itemize}

Mit diesen Werkzeugen, die uns entweder sehr vertraut oder mit denen wir noch einstiegen, haben wir unsere eigene Wege gefunden, wie die mit einander kombiniert werden, um einsere Projektidee umsetzen zu können und Erfolg zu erzielen. 

\section{Docker}\label{Docker}

Die Verwaltung komplexer Datenbanken, Programmiersprachen, Frameworks und anderer Komponenten beim Erstellen von Anwendungen kann eine Herausforderung darstellen. Es besteht auch die Gefahr von Kompatibilitätsproblemen, insbesondere bei der Arbeit mit verschiedenen Betriebssystemen. Diese Faktoren können die Effizienz und den Erfolg Ihrer Arbeit beeinträchtigen.
Glücklicherweise bietet Docker eine elegante Lösung für diese Probleme. Durch die Nutzung von Docker können Anwendungen in Container-Umgebungen erstellt und verwaltet werden. Dies ermöglicht nicht nur eine saubere Trennung von Ressourcen, sondern erleichtert auch die Portabilität zwischen verschiedenen Umgebungen erheblich.
Docker vereinfacht die Entwicklung und macht sie effektiver, indem es viele zeitaufwändige Konfigurationsaufgaben automatisiert. Als Open-Source-Plattform bietet es eine flexible Lösung für Entwickler, um Anwendungen in einer sicheren Sandbox zu entwickeln.
In einer Welt, in der Effizienz und Skalierbarkeit von größter Bedeutung sind, ist Docker ein unverzichtbares Werkzeug für Entwickler. Durch die Nutzung von Containertechnologie können Sie sich auf das konzentrieren, was wirklich zählt: die Entwicklung großartiger Anwendungen.

\subsection{Was ist ein Docker?}

Der Begriff „Docker" wird vielfältig verwendet und kann sich auf das Open-Source-Community-Projekt, Tools aus dem Open Source-Projekt oder direkt auf das Unternehmen Docker Inc. beziehen

Docker ist eine revolutionäre Open-Source-Plattform, die die Art und Weise verändert hat, wie wir Anwendungen entwickeln, testen und bereitstellen. Sie bietet eine Sandbox-Umgebung, in der Entwickler Anwendungen in Containern erstellen können ,leichte, virtualisierte Umgebungen, die eine effiziente Nutzung von Ressourcen ermöglichen. 
Obwohl Container als Konzept schon seit 1979 existieren, hat Docker den Zugang zu ihnen drastisch vereinfacht. Durch Docker können Entwickler Anwendungen lokal oder auf Produktionsservern erstellen, testen und implementieren, ohne sich um Kompatibilitätsprobleme oder Abhängigkeiten zu sorgen.
Seit der Veröffentlichung von Docker 1.0 im Jahr 2014 hat sich die Nutzung von Containern als Standardpraxis für Einzelpersonen und Unternehmen etabliert. Große Unternehmen wie Netflix, Target und Adobe setzen Docker erfolgreich ein, und die Plattform verzeichnet mittlerweile mehr als 13 Millionen Nutzer weltweit. 
Docker hat die Art und Weise, wie wir Software entwickeln und bereitstellen, revolutioniert und ermöglicht es Entwicklern, effizienter und agiler zu arbeiten. Mit Docker können Anwendungen schnell und konsistent zwischen verschiedenen Umgebungen bereitgestellt werden, was zu einer beschleunigten Entwicklungszeit und einer verbesserten Skalierbarkeit führt. 

\subsubsection{Wie funktioniert Docker ?}


Die Docker-Technologie verwendet den Linux-Kernel und seine Funktionen, einschließlich Cgroups und Namespaces, um Prozesse zu unterscheiden, damit sie separat ausgeführt werden können. Der Zweck des Containers ist die Unabhängigkeit – die Fähigkeit, mehrere Prozesse und Apps unabhängig voneinander auszuführen. Dadurch wird die Nutzung Ihrer Infrastruktur optimiert und gleichzeitig die Sicherheit gewährleistet, die durch die Arbeit mit verschiedenen Systemen entsteht. 
Container-Tools wie Docker verwenden ein imagebasiertes Deployment-Modell. Dies macht es einfacher, eine Anwendung oder eine Reihe von Services mit allen Abhängigkeiten in verschiedenen Umgebungen zu nutzen. Darüber hinaus automatisiert Docker das Deployment der Anwendung (oder einer Kombination von Prozessen, aus denen eine Anwendung besteht) innerhalb dieser Containerumgebung.
Diese Programme verwenden Linux-Container, was Docker unglaublich einfach und einzigartig macht.
Wie funktioniert Docker ?


Die Docker-Technologie verwendet den Linux-Kernel und seine Funktionen, einschließlich Cgroups und Namespaces, um Prozesse zu unterscheiden, damit sie separat ausgeführt werden können. Der Zweck des Containers ist die Unabhängigkeit – die Fähigkeit, mehrere Prozesse und Apps unabhängig voneinander auszuführen. Dadurch wird die Nutzung Ihrer Infrastruktur optimiert und gleichzeitig die Sicherheit gewährleistet, die durch die Arbeit mit verschiedenen Systemen entsteht. 
Container-Tools wie Docker verwenden ein imagebasiertes Deployment-Modell. Dies macht es einfacher, eine Anwendung oder eine Reihe von Services mit allen Abhängigkeiten in verschiedenen Umgebungen zu nutzen. Darüber hinaus automatisiert Docker das Deployment der Anwendung (oder einer Kombination von Prozessen, aus denen eine Anwendung besteht) innerhalb dieser Containerumgebung.
Diese Programme verwenden Linux-Container, was Docker unglaublich einfach und einzigartig macht.


\subsection{Was ist ein Netzwerk?}

Ein Netzwerk in Docker ist eine virtuelle Umgebung, die es Containern ermöglicht, miteinander zu kommunizieren. Es ermöglicht die Verbindung von Containern und ermöglicht es ihnen, Ressourcen und Informationen auszutauschen. Netzwerke in Docker können verwendet werden, um verschiedene Container miteinander zu verbinden und sie in einer isolierten Umgebung zu betreiben. 

\subsection{Was ist ein Image?}

Images sind schreibgeschützte Templates mit Anweisungen zum Erstellen eines Containers. Ein docker Image erstellt Container zur Ausführung auf der Docker-Plattform. ein Docker-Image führt Code in einem Docker-Container aus und es ist möglich mehrere Container aus demselben Image zu erstellen.

\subsubsection{Wie erstellt man ein Docker-Image ?}

Um ein Docker-Image zu erstellen, verwendet man den \texttt{docker image build} Befehl zusammen mit weiteren Optionen zur Spezifikation des Build-Kontexts und des Tags des Images. Zum Beispiel:

\begin{verbatim}
docker image build --progress=plain -t image-fancy context
\end{verbatim}

Dieser Befehl erstellt ein Image mit dem Namen \texttt{image-fancy} aus dem Build-Kontext \texttt{context}, wobei \texttt{--progress=plain} eine einfachere Ausgabe während des Build-Prozesses erzeugt.

\subsection{Was ist ein Container ? }

Ein Container ist eine Abstraktion aud fer App-Ebene, die Code und Abhängigkeiten zusammenfasst. Mehrere Container könnenauf demselben Computer ausgeführt werden und den Betriebssystemkernel mit anderen Containern teilen, die jeweils als isolierte Prozesse im Benutzerbereich ausgeführt weden. Container nehmen weniger Platz ein als VMs und können mehr Anwendungen verarbeiten.

\subsubsection{Wie erstellt und startet man einen Container?}

Ein Container wird basierend auf einem Docker-Image erstellt und gestartet. Das folgende Beispiel zeigt, wie ein Container mit dem Namen \texttt{services} erstellt und konfiguriert wird:

\begin{verbatim}
docker container create \
    --name "services" \
    --hostname "services" \
    --network mynet \
    --volume "$PWD/www/:/var/www/html" \
    --volume "$PWD/log/:/log" \
    --publish 80:80 \
    --publish 8080:8080 \
    "image-fancy"
\end{verbatim}


Dieses Kommando erstellt einen Container, der zum \texttt{mynet} Netzwerk gehört, Volumes für Webinhalte und Logdateien einbindet und Ports 80 sowie 8080 veröffentlicht.

Um einen Docker-Container zu starten, verwendet man den Befehl \texttt{docker container start} gefolgt vom Namen oder der ID des Containers, den man starten möchte. Dieser Befehl ändert den Zustand des Containers von gestoppt zu laufend.

Wir haben einen Docker-Container mit dem Namen \texttt{services} erstellt. Um diesen Container zu starten, der Befehl dazu lautet:

\begin{verbatim}
docker container start services
\end{verbatim}

Dieser Befehl sucht nach einem Docker-Container, der \texttt{services} heißt, und startet ihn. Sobald der Container gestartet ist, läuft er im Hintergrund weiter, und die in ihm enthaltene Anwendung beginnt mit ihrer Ausführung basierend auf den beim Erstellen des Containers festgelegten Konfigurationen und Einstellungen.

\subsubsection{Überprüfung des Container-Status}

Um zu überprüfen, ob der Container erfolgreich gestartet wurde und läuft, kann man den Befehl \texttt{docker container ls} verwenden, der alle laufenden Container auflistet. Wenn Sie auch gestoppte Container sehen möchten, können Sie \texttt{docker container ls --all} hinzufügen, um eine vollständige Liste zu erhalten.

\subsection{Wie stoppt und entfernt man einen Container?}

Zum Stoppen und Entfernen eines Containers verwendet man die Befehle \texttt{docker container stop} und \texttt{docker container rm}. Das folgende Skript demonstriert diesen Prozess für einen Container namens \texttt{services}:

\begin{verbatim}
if docker container ls | grep -q "\bservices$"; then
    docker container stop services
fi

if docker container ls --all | grep -q "\bservices$"; then
    docker container rm services
fi
\end{verbatim}

Zusätzlich entfernt das Skript veraltete SSH-Konfigurationen für den Container aus \texttt{\~/.ssh/docker\_config}.

\subsection{Was sind die Vor - und Nachteile von Docker ? }

\subsubsection{Vorteile}

Docker bietet eine Vielzahl von Vorteilen, die es zu einem beliebten Werkzeug in der Softwareentwicklung und im IT-Betrieb gemacht haben:

\begin{itemize}
    \item \textbf{Portabilität:} Einmal erstellte Docker-Images können über verschiedene Umgebungen hinweg ohne Änderungen ausgeführt werden. Dies erleichtert die Bereitstellung und Migration von Anwendungen erheblich.
    \item \textbf{Konsistenz:} Docker-Container bieten eine konsistente Umgebung für die Anwendung, unabhängig davon, wo der Container ausgeführt wird. Dies reduziert das Problem "es funktioniert auf meinem Rechner nicht", indem Diskrepanzen zwischen Entwicklung, Test und Produktion minimiert werden.
    \item \textbf{Isolation:} Container sind voneinander isoliert und verpacken ihre eigene Software, Bibliotheken und Konfigurationen. Dies verbessert die Sicherheit und vereinfacht Abhängigkeitsmanagement.
    \item \textbf{Ressourceneffizienz:} Docker nutzt die Ressourcen des Host-Betriebssystems effizienter als traditionelle Virtualisierungstechnologien, was zu einer besseren Hardwareauslastung führt.
    \item \textbf{Schnelle Bereitstellung:} Container können innerhalb von Sekunden gestartet und gestoppt werden, was die Bereitstellungszeiten verkürzt und eine schnelle Skalierung ermöglicht.
    \item \textbf{Entwicklungseffizienz:} Docker unterstützt Entwickler mit der Möglichkeit, lokale Umgebungen schnell einzurichten, was die Entwicklungszyklen beschleunigt.
    \item \textbf{Mikroservice-Architekturen:} Docker eignet sich hervorragend für die Entwicklung und den Betrieb von mikroservice-basierten Anwendungen, indem es die Unabhängigkeit und Isolation der Services fördert.
\end{itemize}

\subsubsection{Nachteile}

Trotz seiner vielen Vorteile hat Docker auch einige Nachteile, die berücksichtigt werden sollten:

\begin{itemize}
    \item \textbf{Sicherheitsrisiken:} Die Isolation zwischen Containern ist nicht so stark wie bei vollständig virtualisierten Umgebungen, was potenzielle Sicherheitsrisiken birgt. Es ist wichtig, Sicherheitspraktiken wie das Prinzip der geringsten Privilegien und regelmäßige Updates zu befolgen.
    \item \textbf{Komplexität:} Die Verwaltung von Docker-Containern und -Orchestrierung, insbesondere in großem Maßstab, kann komplex sein und erfordert ein gutes Verständnis von Docker-Netzwerken, Volumes und Sicherheitskonfigurationen.
    \item \textbf{Speicher- und Datenmanagement:} Persistente Daten und deren Verwaltung können in containerisierten Anwendungen eine Herausforderung darstellen, besonders wenn es um Zustandsbehaftete Dienste wie Datenbanken geht.
    \item \textbf{Plattformabhängigkeiten:} Obwohl Docker auf Portabilität ausgelegt ist, können bestimmte Images aufgrund von Systemabhängigkeiten (z.B. Kernel-Versionen) nicht auf allen Plattformen ausgeführt werden.
    \item \textbf{Performance-Überlegungen:} Während Container im Allgemeinen effizient sind, können spezifische Workloads oder Konfigurationen in Containern im Vergleich zu nativen Ausführungen oder anderen Virtualisierungstechnologien Leistungseinbußen erleiden.
    \item \textbf{Lernkurve:} Die umfangreichen Funktionen und das Ökosystem von Docker können für Neueinsteiger überwältigend sein und erfordern Zeit und Ressourcen, um effektiv genutzt zu werden.
\end{itemize}

\subsection{Wie haben wir Docker verwendet?}

In unseren Skripten haben wir Docker verwendet, um eine isolierte und konsistente Umgebung für unsere Anwendungen zu schaffen. Die Skripte automatisieren den Prozess des Builds, der Bereitstellung und des Managements von Containern. Hier einige Beispiele:

\begin{itemize}
    \item \textbf{Image-Erstellung:} Wir nutzen den \texttt{docker image build} Befehl, um Docker-Images aus unseren Anwendungsquellen zu erstellen. Dies ermöglicht eine einheitliche und reproduzierbare Umgebung für die Entwicklung und das Deployment unserer Anwendung.
    \item \textbf{Container-Management:} Unsere Skripte automatisieren das Erstellen, Starten, Stoppen und Entfernen von Containern. Dies vereinfacht die Verwaltung der Lebenszyklen unserer Anwendungen erheblich.
    \item \textbf{Netzwerk und Volumen:} Wir konfigurieren Netzwerke und binden Volumes ein, um die Kommunikation zwischen Containern zu ermöglichen und persistente Daten zu verwalten. Dies ist entscheidend für komplexe Anwendungen, die aus mehreren Diensten bestehen und Zustandsdaten speichern müssen.
\end{itemize}


\section{Ergebnis}\label{Ergebnis}

%Hier später bei Fortschritt irgendwas hinzufügen

Bei dieser Ausarbeitung haben wir sehr viel Wert darauf gelegt, eine fundierte Analyse anhand unsere Gelernten aus der Vorlesung abzugeben. AUs diesem Grund haben wir nicht nur irgendeinen Code aus der Vorlesung kopiert und eingefügt, sondern, wir haben das Wissen aus der Vorlesung zu unserem Nutzen gemacht und daraus erfolgten verschiedene Analysen mit einem und zwei und noch mehr Containern, damit wir effektiv zeigen können, wie eine Dockerumgebung verwendet werden kann.

\subsection{Analyse mit zwei laufenden Containern}

%Hier sind wichtige Punkte

%was haben wir gecodet?
%wie haben wir das gecodet bzw. implementiert?
%was haben wir uns dabei gehofft?
%auf welche Ergebnisse kamen wir?
%Was haben wir dabei gelernt?
Basierend auf unsere Idee, ein Java-Servlet mit verschiedenen Dockerkonstellationen laufen zu lassen, haben wir verschiedenen Sachen implementiert. Bei der ersten sowie bei den anderen Anaylse stand im Vordergrund die Vorbereitung der Arbeitsumgebung. Wir mussten uns eine geeignete Dockerumgebung erstellen. Hierfür haben wir zuerst die vorbereitete Umgebung aus der Vorlesung verwendet, aber vorher haben wir die noch ein wenig angepasst, sodass es unsere Vorstellungen entsprechen kann.\\

Der Ordner, den wir hierfür  ausgewählt haben, war \begin{verbatim} /docker-fancy-tomcat-split-sql \end{verbatim}, weil wir gefunden haben, dass er leichter wäre noch mehrfach aufzuteilen. Für dieses erste Experiment waren die Docker nur in  zwei Container jeweils \texttt{work und services}, wobei \textbf{work} unsere Arbeitsumgebung darstellt. Dort wird nur gearbeitet und im Anschluss daran später zum Container \textbf{services} deployt. Im Container \textbf{servcies} laufen im Gegenteil zu \textbf{work} all unsere Dienste, wie zum Beispiel: \texttt{Redis, MariaDB, Tomcat, ein ssh-Server und ein Apache-Server}. Beide Container werden durch einen ssh-Server verbunden und es ist so eingerichtet worde, dass man sich von einem Container zu einem anderen passwortlos bewegen kann.\\

Der Grund, warum wir nicht mit einem Container gearbeitet haben ist leicht zu erklären. Es liegt einfach daran, dass wir vom Vorteil solch einer Umgebung Gebrauch machen wollten, und zwar die Isolation. Die Prozesse auf eine einem Docker finden nämlich isoliert statt, was zu verbesserten Leistungen führt. Noch wichtig zu der Isolation wäre die erhöhte Flexibilität. Nehmen wir als Beispiel einen Fall, den wir bei uns hatten mit einem nicht aktuellen Container hatten. In so einer Situation ließ sich dieser mit wenig Aufwand durch einen aktuellen Container, der unserer Anforderungen entsprach, tauschen.\\

\subsubsection{Aufbau der Image: \texttt{fancy}}

\subsubsection{Aufbau unseres lokalen Netzwerkes: \textt{mynet}}

\subsubsection{Aufbau der Container: \texttt{work und service}}

Wie oben erwähnt haben wir uns in diesem ersten Fall mit zwei laufenden Container auseinandergesetzt. und zwar \texttt{work} und \texttt{service}. Betrachten wir nun mal den Aufbau dieser beiden Container.

Wir beginnen erstmal mit \texttt{work}. In diesem Container wurde hauptsächlich gearbeitet, aus diesem Grund ist seine Struktur nur grundlegend und enthält nur einen ssh-Dienst, der beim Starten vom Container ebenfalls gestartet wird. Hierfü stand schon das bash-Skript \texttt{myinit-work.sh} zur Verfügung. dieser Skript ist dafür zuständig einen User anzulegen und den ssh-Dienst zu starten. natürlich gibt es auch einen Aschnitt bzw eine Funktion im Skript, der/die sicherstellt, dass der gestartete Dienst auch entsprechend gestoppt, wird, wenn der Container beendet wird.


\defverbatim\CodeOne{
  \begin{minted}{bash}
    # start services
    {
      /startup/create-user.sh
      #pw=$(grep "^containerpassword=" /startup/config.txt|cut -d'=' -f2)
      service ssh start
      #rm /startup/*
      echo ready >> /log/state-work.txt
    } >>$log

  \end{minted}
  \captionof{listing}{Starten von Diensten in \texttt{\textbf{work}}}
}

Zum Starten vom Container selbst wird ein anderes Skript names \texttt{start-work.sh} verwendet. Dieses Skript sorgt einfach dafür, dass der work-Container gestartet wird, aber mit zusätzlichen Optionen, wie zum Beispiel, dass es in einem lokal erstellten Netzwerk läuft, vorbereitetete Konfigurationen übernimmt, die sich schon vorab in dem Verzeichnis Startup in \texttt{context/} befinden, und auch zu vorhandenen bekannten Hots hinzugefügt wird. Dadurch wird es möglich sich in diesem Container per ssh zu bewegen, statt auf der Kommandozeile \begin{verbatim} docker container exec -ti work bash \end{verbatim} eingeben zu müssen. Zu den wichtigsten Sachen die work beim Starten kopiert werden, zählen wir, Java-Programme, die Teil unseres Servelt sind.
\defverbatim\codeOne{
  \begin{minted}
    ssh $containername :
    scp -qr javawebdemo $containername:
    scp -qr brotundbutter-swe $containername:
    scp -qr unittester $containername:
  \end{minted}
}

Da Work nur der Arbeitsbereich ist und dort nur gecode wird, ist das tatsächlich so, dass man zum Ausführen von Programmen, die zu dem anderen Container deployt. dort laufen alle notwendige Dienste zur korrekten Ausführung der Programme. Die werden ebenfalls in von einem Skript namens \texttt{myinit-services.sh} gestartet und beenden. Es gibt unter anderem solche Dienst wie Redis, Tomcat, MariaDB und einen SSH-Dienst.

\\

\subsubsection{Hoffnung und Erwartungen}


Im Laufe des Semesters konnte wir durch unsere Teilnahme an den Veranstaltungen sehr viele Ideen zum Testen haben. Das haben wir versucht in diesem projekt wiederzuspiegeln. Darauf haben wir stets abgezielt, die beigebrachten Konzepten anzuwenden. Wir wollten zum Beispiel Networking anwenden, was ein Zusammenschluss von zwei oder mehr Computern oder anderen, elektronischen Geräten, der den Austausch von Daten und die Nutzung gemeinsamer Ressourcen ermöglicht \cite{ionos}. Wenn bei uns jeder Container einzeln als Computer betrachtet wird, können wir auch von Networking reden. In einem Netz hat jeder Computer eine eigene IP-Adresse, mit der er über das Netz angesprochen werden kann. Dies haben wir uns bei der testphase zur Nutze gemacht, um wiederholt Anfragen an unseren Programmen zu senden.

\\


Noch wärhend der Veranstaltung wurden wir in die POSIX-Standard eingeführt. Was wir dazu wissen ist, POSIX heißt (aus dem Englischen) Portable Operating System Interface und ist eine gmeinesam von IEEE und der Open Group für Unix entwickelte standardisierte Programmierschnittstelle, welche die Schnittstelle zwischen Anwendungssoftware und Betriebssystem darstellt. Es ist zu einer Norm geworden, die bei allen Unic-Betriebssystemen verwendet wird. Dabei ist ein betriebssystem wie Windows ausgeschlossen, denn dieses System hat seine eigene Standart. Früher hat jedoch Microsoft versucht, sich an diese Standards zu halten, sind aber später davon rausgegangen.

\\



\subsection{Analyse mit drei laufenden Containern}

\subsection{Analyse mit vier laufenden Containern}

\section{Lasttest}\label{Lasttest}


\subsection{Was haben wir analysiert?}



\subsection{Wie haben wir das analysiert?}



\subsection{Endergebnis}

\input{src/reflexion.tex}
\input{src/fazit.tex}
\newpage

% Literaturverzeichnis anzeigen
\ohead{Literaturverzeichnis} % Korrektur für Header 
\phantomsection
\addcontentsline{toc}{section}{Literaturverzeichnis}
\renewcommand\refname{Literaturverzeichnis}
\printbibliography
\newpage



% Abkürzungsverzeichnis anzeigen
%\ohead{Abkürzungsverzeichnis} % Korrektur für Header 
%\section*{Abkürzungsverzeichnis}
%\input{src/basic_structure/abkuerzungen.tex}
%\addcontentsline{toc}{section}{Abkürzungsverzeichnis}
%\newpage



% Selbstständigkeits Erklärung
\phantomsection
\addcontentsline{toc}{section}{Selbstständigkeitserklärung}

% Header für Erklärung
\ohead{Selbstständigkeitserklärung}

% Input Erklärung
\input{src/basic_structure/erklaerung.tex}

% Leere Abschlussseite
%\newpage
%\thispagestyle{empty} % erzeugt Seite ohne Kopf- / Fusszeile
%\mbox{}

\end{document}
