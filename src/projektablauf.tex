\section{Projektablauf}\label{Projektablauf}

\subsection{Vorstellung der Projektidee}

\subsection{Projektorganisation und Teamarbeit}


Für unsere Auseinandersetzung zum Thema "\/Analyse verschiedener Dockerkonstellationen"\/ im Rahmen der Veranstaltung \/"Vernetzte Systeme"\/ haben wir uns zegielt für einen strukturierten Projektplan, der sich an agilen Methoden orientiert. Zum Entwerfen und Fertigstellung haben wir einen Sprint von drei Wochen gemacht inklusiv das Verfassen vom Bericht. Regelmäßig haben wir uns alle zwei Tage getroffen und da wurde immer über die Fortschritte gesprochen und neue Aufgabe verteilt. Die ersten Tage habe wir erstmal dürber ausgetauscht, was wir machen wollen und wie wir es tun möchten?

\\

Die Implementierung haben wir in mehreren iterativen Phasen unterteilt, wobei das Kanban-System zur Visualisierung und Verwaltung des Arbeitsflusses eingesetzt wurde. Hierfür haben wir die Webanwendung Trello(Verlinkung fehlt)  verwendet. In der App haben wir über einen Projekt-Dashboard verfügt, auf dem unsere Aufgaben in drei Kategorien aufgelistet waren wie:

\begin{itemize}
\item Zu erledigen
\item In Arbeit
\item Erledigt
\end{itemize}

Dies ermöglichte eine flexible Anpassung an wechselnde Anforderungen und eine kontinuierliche Verbesserung der Implementierung. Die agilen Prinzipien unterstützten uns dabei, schnell auf Herausforderungen zu reagieren und eine effektive Implementierung sicherzustellen.
