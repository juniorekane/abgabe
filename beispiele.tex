% Beispiel für Bildintegration

\begin{figure}[!ht]
	\centering
	\includegraphics[width=0.52\textwidth]{src/abbildungen/deckblatt.png}
	\caption{Deckblatt}
	\label{picturedeckblatt}
\end{figure}

% Beispiel für einen Schaltplan 

\begin{figure}
  \centering
  \begin{circuitikz}
    \draw (0,80) node[vcc]    (vcc) {+ \SI{5}{\volt}};
    \draw (0,60) node (d) {};
    \draw (0,30) node (r3) {};
    \draw (0,15) node (r4) {};
    \draw (0,0)  node[rground] (gnd) {};
    \draw (vcc) to[leDo] (d.center);
    \draw (d.center)  to[R=$R_3$,a={\SI{1}{\kilo\ohm}}]  (r3.center);
    \draw (r3.center)  to[R=$R_4$,a={\SI{1}{\kilo\ohm}}]  (r4.center);
    \draw (r4.center)  to  (gnd);
  \end{circuitikz}
  \caption[Schaltung]{Schaltung}
  \label{fig:Schaltung}
\end{figure}



% Beispiel für eingerücktes Zitat (länger als drei Zeilen)

\begin{quote}
	\footnotesize{
		Er hörte leise Schritte hinter sich.
		Das bedeutete nichts Gutes.
		Wer würde ihm schon folgen, spät in der Nacht
		und dazu noch in dieser engen Gasse mitten
		im übel beleumundeten Hafenviertel?
		Gerade jetzt, wo er das Ding seines Lebens gedreht hatte
		und mit der Beute verschwinden wollte~\autocite[postnote]{kürzel}!}
\end{quote}


% Beispiel für Formeln

\begin{quote}
  Die Funktion F: $\mathbb{R} \rightarrow$ [0,1] mit $F(t) = P (X \le t)$ heißt Verteilungsfunktion von $X$. vgl. \cite[S.55]{mf2005}
\end{quote}

% Beispiel für eine Tabelle 

\begin{longtblr}[caption={\LaTeX~Schriftgrößen}, label={schriftgrößen} ]{colspec = {X[l,m] X[2,c,m]}, rowhead=1, width=0.2\textwidth}\toprule
	Befehl                      & Ergebnis                    \\ \midrule
	\textbackslash tiny         & \tiny{winzig}               \\ \cmidrule{1-2}
	\textbackslash scriptsize   & \scriptsize{klein}          \\ \cmidrule{1-2}
	\textbackslash footnotesize & \footnotesize{etwas größer} \\ \cmidrule{1-2}
	\textbackslash small        & \small{noch etwas größer}   \\ \cmidrule{1-2}
	\textbackslash normalsize   & \normalsize{normal}         \\ \cmidrule{1-2}
	\textbackslash large        & \large{groß}                \\ \cmidrule{1-2}
	\textbackslash Large        & \Large{größer}              \\ \cmidrule{1-2}
	\textbackslash LARGE        & \LARGE{noch größer}         \\ \cmidrule{1-2}
	\textbackslash huge         & \huge{riesig}               \\ \cmidrule{1-2}
	\textbackslash Huge         & \LARGE{noch riesiger}       \\ \bottomrule
\end{longtblr}

% Beispiel für Listing

\begin{code}{Ein Listing}{einlisting}
	\begin{minted}{bash}
	for i in {1..10}; do 
		echo "$i"
	done	
	\end{minted}
\end{code}

% Beispiel: Verweis auf eine Abbildung

Abbildung~\ref{labelname}

% Beispiel Zitat

~\autocite[postnote]{kürzel}

